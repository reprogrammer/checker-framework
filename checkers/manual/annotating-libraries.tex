\htmlhr
\chapter{Annotating libraries\label{annotating-libraries}}

When annotated code uses an unannotated library, a checker may issue warnings.
As described in Section~\ref{unannotated-code}, the best way to correct
this problem is to add annotations to the library.  (Alternately, you can instead
suppress all warnings related to an unannotated library by use of the 
\code{-AskipClasses} command-line option; see
Section~\ref{suppressing-warnings}.)  If you have source code for the
library, you can easily add the annotations.
This section tells you
how to add annotations to a library for which you have no source code,
because the library is distributed only in binary (\code{.class} or
\code{.jar}) form.  This section is also useful if you do not wish to edit the
library's source code.

The Checker Framework distribution contains annotations
for popular libraries, such as the JDK\@.
If you annotate additional libraries, please share them with us so that we
can distribute the annotations with the Checker Framework; see
Section~\ref{reporting-bugs}.


You can determine the correct annotations for a library either
automatically by running an inference tool, or manually by reading the
documentation.  Presently, type inference tools are available for the
Nullness (Section~\ref{nullness-inference}) and Javari
(Section~\ref{javari-inference}) type systems.

You can make the annotations known to the JSR 308 compiler (and thus
to the checkers) in two ways.

\begin{itemize}

\item You can use the stub class generation tool to create a ``stub
  file'' containing classes with no method bodies,
  and annotate the stub classes
  file.  Then, you can supply the stub files to the checker when
  compiling/checking your program.
  Section~\ref{stub} describes how to use the stub class generation
  tools.

\item You can annotate the compiled
  \code{.jar} or \code{.class} files using the annotation file utilities
  (\myurl{http://types.cs.washington.edu/annotation-file-utilities/}).
  First, express the annotations textually as an annotation index file, and
  then the tools insert them in the compiled library class files.
  See the Annotation File Utilities documentation for full details.

\end{itemize}


\section{Using stub classes\label{stub}\label{stub-creating-and-using}}

A stub file contains ``stub classes'' that contain annotated signatures.  A
checker uses those annotated signatures at compile time, instead of or in
addition to annotations that appear in the library.

Section~\ref{stub-creating} describes how to create stub classes.
Section~\ref{stub-using} describes how to use stub classes.
These sections illustrate stub classes via the example of creating a \code{@\refclass{interning/quals}{Interned}}-annotated
version of \code{java.lang.String}.  (You don't need to repeat these steps,
since such a stub class is already included in the Checker Framework
distribution; see file \code{checkers/src/checkers/interning/jdk.astub}, which
is reproduced in Section~\ref{stub-format}.)

% First, you must install the skeleton class generator
% (Section~\ref{skeleton-installing}).

\subsection{Creating a stub file\label{stub-creating}}

\begin{enumerate}

\item
  Create a stub file by running the stub class generator.  (\<checkers.jar>
  must be on your classpath.)

\begin{Verbatim}
  cd nullness-stub
  java checkers.util.stub.StubGenerator java.lang.String > String.astub
\end{Verbatim}

  Supply it with the fully-qualified name of the class for which you wish to
  generate a stub class.  The stub class generator prints the
  stub class to standard out, so you may wish to redirect its output to a
  file.

\item
  Add import statements for the annotations.  So you would need to
add the following import statement at the beginning of the file:

\begin{Verbatim}
  import checkers.interning.quals.Interned;
\end{Verbatim}

\item
  Add annotations to the stub class.  For example, you might annotate
  the \ahref{http://java.sun.com/javase/6/docs/api/java/lang/String.html#intern()}{\<String.intern()>} method as follows:

\begin{Verbatim}
  @Interned String intern();
\end{Verbatim}

  You may also remove irrelevant parts of the stub file; see
  Section~\ref{stub-format}.

\end{enumerate}


\subsection{Using a stub file\label{stub-using}}

  When you run \code{javac} with a given checker/processor, you can specify
  a list of the stub files or directories using
  \code{-Astubs=\emph{file\_or\_path\_name}}.  The stub path entries
  are delimited by
  \<File.pathSeparator> (`\<:>' for Linux and Mac, `\<;>' for Windows).
  When you supply a stub directory, the checker only considers the enclosed
  stub files whose names end with \code{.astub}.

  The \code{-Astubs} argument causes the Checker Framework to read annotations
  from annotated stub classes in preference to the unannotated original
  library classes.

%BEGIN LATEX
\begin{smaller}
%END LATEX
\begin{Verbatim}
  javac -processor checkers.interning.InterningChecker -Astubs=String.astub:stubs MyFile.java MyOtherFile.java ...
\end{Verbatim}
%BEGIN LATEX
\end{smaller}
%END LATEX


\subsection{Stub file format\label{stub-format}}

The stub file format is designed for simplicity, readability, and
compactness.  It reads like a Java file but contains only the
necessary information for type checking.

As an illustration, the stub file for the Interning type system
(Chapter~\ref{interning-checker}) is as follows.  This file appears as
\code{checkers/src/checkers/interning/jdk.astub} in the Checker Framework
distribution.

\begin{Verbatim}
  import checkers.interning.quals.Interned;

  package java.lang;

  // All instances of Class are interned.
  @Interned class Class<T> { }

  class String {
    // The only interning-related method in the JDK.
    @Interned String intern();
  }
\end{Verbatim}


You can use a regular Java file as a stub file.  Every valid Java file is a
valid stub file.  However, you can omit
information that is not relevant to pluggable type-checking; this makes the
stub file smaller and easier for people to read and write.  You can also
put annotated signatures for multiple classes in a single stub file.


The stub file format is allowed to differ from Java source code in the
following ways:
\begin{description}

\item{\textbf{Method bodies:}}
  The stub class does not require method bodies for classes; any method
  body may be replaced by a semicolon (\code{;}), as in an interface or
  abstract method declaration.

\item{\textbf{Method declarations:}}
  You only have to specify the methods that you need to annotate.
  Any method declaration may be omitted, in which case the checker reads
  its annotations from the library.  (If you are using a stub class, then
  typically the library's version is unannotated.)

\item{\textbf{Declaration specifiers:}}
  Declaration specifiers (e.g., \<public>, \<final>, \<volatile>)
  may be omitted, since they have nothing to do with types.

\item{\textbf{Import statements:}}
  The only required import statements are the ones to import type
  annotations.  Such imports must be at the beginning of the
  file.  Other import statements are optional.

\item{\textbf{Multiple classes and packages:}}
  The stub file format permits having multiple classes and packages.
  The packages are separated by a package statement:
  \<package my.package;>.  Each package declaration may occur only once; in
  other words, all classes from a package must appear together.

\end{description}


\subsection{Known problems}


The Checker Framework stub file reader has several limitations:

\begin{itemize}
\item
  % Still a problem as of 9/2/2009.
  It does not handle \code{enum}s.
\item
  % Still a problem as of 9/2/2009.
  It only handles type annotations, not declaration annotations (e.g. IGJ's \<Assignable>).
\end{itemize}


\subsection{Style tips for stub files}

Every Java file is a stub file.  If you have access to the Java file, then
it is usually best to use the Java file as the stub file, without removing
any of the parts that the stub file format permits you to.  Just add
annotations to the full source code.  This approach retains the original
documentation and source code, making it easier for a programmer to
double-check the annotations.  It also enables creation of diffs, easing
the process of upgrading when a library adds new methods.  And, the
annotations are in a format that the library maintainers can even
incorporate.

The downside of this approach is that the stub files are larger.  This can
slow down parsing.  Furthermore, a programmer must search the stub file
for a given method rather than just skimming one or two pages of signatures.

If you do not have access to the library source code, then you can create a
stub file from the Javadoc or the class file, and then annotate it.



% Label "skeleton" is for old links from the Javarifier manual, to prevent
% them from being broken links.

\section{Using distributed annotated JDKs\label{skeleton-using}\label{skeleton}}

The Checker Framework distribution contains
annotated JDKs at the path \<checkers/jdk/[checker-name]/jdk.jar>.

% Skeleton classes are inferior to stub classes for two reasons.  First,
% skeleton files must be on the classpath during compilation but must
% \emph{not} be on the classpath during execution; this is inconvenient and
% error-prone.  Second, the skeleton files contain incorrect values for
% certain static final fields.  These incorrect values can lead to
% run-time problems unless the Java code is re-compiled without the skeleton
% classes after type-checking is complete.


\begin{enumerate}

\item
  When you run \code{javac}, add a \code{-bootclasspath/p:} argument to indicate
  where to find the annotated jdk classes.
  Supply \code{-bootclasspath/p} in addition to whatever other arguments you
  usually use, including \code{-classpath}.

  The \code{-bootclasspath/p:} argument causes the compiler to read annotations
  from annotated jdk classes in preference to the unannotated original
  library classes.

%BEGIN LATEX
\begin{smaller}
%END LATEX
\begin{Verbatim}
  javac -processor checkers.nullness.NullnessChecker -Xbootclasspath/p:checkers/jdk/nullness/jdk.jar my_source_files
\end{Verbatim}
%BEGIN LATEX
\end{smaller}
%END LATEX

\item
  Run the compiled code as usual.  The annotated JDK does not need to be
  in your classpath.

\end{enumerate}


% \section{Installing the skeleton class generator\label{skeleton-installing}}
%
% Source code for the skeleton class generator tool is included in the
% Checker Framework
% distribution, but because the tool has additional dependencies, the provided
% build script does not build the tool by default.
%
% Follow these steps to install the skeleton class generator:
%
% \begin{enumerate}
%
% \item
%   Install the annotation file utilities, using the instructions at
%   \myurl{http://types.cs.washington.edu/annotation-file-utilities/}.
%   Per those instructions, the \code{annotation-file-utilities.jar} file
%   should be on your classpath.
%
% % TODO This item should become optional; tell people to install the AFU in
% % the right place.
% \item
%   Update the \code{build.properties} file in the Checker Framework distribution so
%   that the \code{annotation-utils.lib} property specifies the location of
%   the \code{annotation-file-utilities.jar} library.
%
% \item
%   Build the skeleton class generator tool by running \code{ant
%     skeleton-util dist} in the \code{checkers} directory.  This updates the
%   \code{checkers.jar} file to contain the skeleton class generator.
%   \code{checkers.jar} should already be on your classpath (see
%   Section~\ref{installation}).
%
% \end{enumerate}


% LocalWords:  plugin utils util dist RuntimeException NonNull TODO AFU enum
% LocalWords:  sourcepath Nullness javac classpath src quals pathSeparator JDKs
% LocalWords:  IGJ's jdk Astubs skipClasses astub AskipClasses toArray
% LocalWords:  CollectionToArrayHeuristics
