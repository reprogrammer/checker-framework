\htmlhr
\chapter{Nullness checker\label{nullness-checker}}

If the Nullness checker issues no warnings for a given program, then
running that program will never throw a null pointer exception.  This
guarantee enables a programmer to prevent errors from occurring when a
program is run.  See Section~\ref{nullness-checks} for more details about
the guarantee and what is checked.

To run the Nullness Checker, supply the \code{-processor
  checkers.nullness.NullnessChecker} command-line option to javac.  For
examples, see Section~\ref{nullness-example}.


\section{Nullness annotations\label{nullness-annotations}}

The Nullness checker uses three separate type hierarchies:  one for nullness,
one for rawness (Section~\ref{raw-partially-initialized}),
and one for map keys (Section~\ref{map-keys})
The Nullness checker has three varieties of annotations:  nullness
qualifiers, nullness method annotations, rawness qualifiers, and map key
qualifiers.

\subsection{Nullness qualifiers\label{nullness-qualifiers}}

The nullness hierarchy contains these qualifiers:

\begin{description}

\item[\code{@\refclass{nullness/quals}{Nullable}}]
  indicates a type that includes the null value.  For example, the type \code{Boolean}
  is nullable:  a variable of type \code{Boolean} always has one of the
  values \code{TRUE}, \code{FALSE}, or \code{null}.

\item[\code{@\refclass{nullness/quals}{NonNull}}]
  indicates a type that does not include the null value.  The type
  \code{boolean} is non-null; a variable of type \code{boolean} always has
  one of the values \code{true} or \code{false}.  The type \code{@NonNull
    Boolean} is also non-null:  a variable of type \code{@NonNull Boolean}
  always has one of the values \code{TRUE} or \code{FALSE} --- never
  \code{null}.  Dereferencing an expression of non-null type can never cause
  a null pointer exception.

  The \<@NonNull> annotation is rarely written in a program, because it is
  the default (see Section~\ref{null-defaults}).

\item[\code{@\refclass{nullness/quals}{PolyNull}}]
  indicates qualifier polymorphism.  For a description of
  \<@\refclass{nullness/quals}{PolyNull}>, see
  Section~\ref{qualifier-polymorphism}.

\item[\code{@\refclass{nullness/quals}{LazyNonNull}}]
  indicates a reference that may be \code{null}, but if it ever becomes
  non-\code{null}, then it never becomes \code{null} again.  This is
  appropriate for lazily-initialized fields, among other uses.  When the
  variable is read, its type is treated as
  \code{@\refclass{nullness/quals}{Nullable}}, but when the variable is
  assigned, its type is treated as
  \code{@\refclass{nullness/quals}{NonNull}}.

  Because the Nullness checker works intraprocedurally (it analyzes one
  method at a time), when a \code{LazyNonNull} field is first read within a
  method, the field cannot be assumed to be non-null.  The benefit of
  LazyNonNull over Nullable is its different interaction with
  flow-sensitive type qualifier refinement (Section~\ref{type-refinement}).
  After a check of a LazyNonNull
  field, all subsequent accesses \emph{within that method} can be assumed
  to be NonNull, even after arbitrary external method calls that have
  access to the given field.

\end{description}

Figure~\ref{fig:nonnull-hierarchy} shows part of the type hierarchy for the
Nullness type system.
(The annotations exist only at compile time; at run time, Java has no
multiple inheritance.)

\begin{figure}
\includeimage{nullness-and-raw}{2.5cm}
\caption{Partial type hierarchy for the Nullness type system.
Java's \<Object> is expressed as \<@Nullable Object>.  Programmers can omit
most type qualifiers, because the default annotation
(Section~\ref{null-defaults}) is usually correct.  Also shown is the
type hierarchy for rawness (Section~\ref{raw-partially-initialized}), which
indicates whether
initialization has completed.  The two type hierarchies are independent but
inter-related.  The Nullness Checker verifies both of these, as well as
another type hierarchy for map keys (Section~\ref{map-key-qualifiers}).}
\label{fig:nonnull-hierarchy}
\end{figure}


\subsection{Nullness method annotations\label{nullness-non-qualifiers}}

The Nullness checker supports several annotations that specify method
behavior.

\begin{description}

\item[\code{@\refclass{nullness/quals}{NonNullOnEntry}}]
  indicates a method precondition:  The annotated method expects the
  specified variables (typically field references) to be non-null when the
  method is invoked.

\item[\code{@\refclass{nullness/quals}{AssertNonNullAfter}}]
\item[\code{@\refclass{nullness/quals}{AssertNonNullIfTrue}}]
\item[\code{@\refclass{nullness/quals}{AssertNonNullIfFalse}}]
  indicates a method postcondition.  With \<@AssertNonNullAfter>, the given
  expressions are non-null after the method returns; this is useful for a
  method that initializes a field, for example.  With
  \<@AssertNonNullIfTrue> and \<@AssertNonNullIfFalse>, if the annotated
  method returns the given boolean value (true or false), then the given
  expressions are non-null.  See Section~\ref{conditional-nullness} and the
  Javadoc for examples of their use.

\item[\code{@\refclass{nullness/quals}{Pure}}]
  indicates that the method has no (visible) side effects.

  Furthermore, if the method is called multiple times with the same
  arguments, then it returns the same result.  For example, suppose that
  the return value of method \code{m} is nullable.  Then this code will
  pass the type-checker:

\begin{Verbatim}
        if (m(arg) != null) {
          m(arg).toString();
        }
\end{Verbatim}

  \noindent
  only if method \code{m} is annotated as \code{@Pure}.

  If a method is pure, then it would be legal to annotate its receiver and
  every parameter as \<@ReadOnly>, in the IGJ (Chapter~\ref{igj-checker}) or
  Javari (Chapter~\ref{javari-checker}) type systems.  The reverse is not
  true, both because the method might side-effect a global variable and
  because the method might not be deterministic.

\item[\code{@\refclass{nullness/quals}{AssertParametersNonNull}}]
  % Indicates a method precondition:  The annotated method expects all of
  % its parameters to be non-null.
  is used for suppressing warnings, in very rare cases.  See the Javadoc for
  details.

\end{description}


\subsection{Rawness qualifiers\label{rawness-qualifiers}}

The Nullness Checker supports rawness annotations that indicate whether
an object is fully initialized --- that is, whether its fields have all
been assigned.

\begin{description}
\item[\code{@\refclass{nullness/quals}{Raw}}]
\item[\code{@\refclass{nullness/quals}{NonRaw}}]
\item[\code{@\refclass{nullness/quals}{PolyRaw}}]
\end{description}

Use of these annotations can help you to type-check more
code.  Figure~\ref{fig:nonnull-hierarchy} shows its type hierarchy.  For
details, see Section~\ref{raw-partially-initialized}.


\subsection{Map key qualifiers\label{map-key-qualifiers}}

The Nullness Checker supports a map key annotation, \code{@\refclass{nullness/quals}{KeyFor}} that indicates whether
a value is a key for a given map --- that is, whether
\code{map.containsKey(value)} would evaluate to \code{true}.

\begin{description}
\item[\code{@\refclass{nullness/quals}{KeyFor}}]
\end{description}

Use of this annotation can help you to type-check more code.  For details,
see Section~\ref{map-keys}.


\section{Writing nullness annotations\label{writing-nullness-annotations}}

\subsection{Implicit qualifiers\label{nullness-implicit-qualifiers}}

As described in Section~\ref{effective-qualifier}, the Nullness checker
adds implicit qualifiers, reducing the number of annotations that must
appear in your code.
For example, enum types are implicitly non-null, so you never need to write
\<@NonNull MyEnumType>.

For a complete description of all implicit nullness qualifiers, see the
Javadoc for \refclass{nullness}{NullnessAnnotatedTypeFactory}.



\subsection{Default annotation\label{null-defaults}}

Unannotated references are treated as if they had a default annotation,
using the NNEL (non-null except locals) rule described below.
A user may choose a different rule for defaults using the
\code{@\refclass{quals}{DefaultQualifier}} annotation; see
Section~\ref{defaults}.

%BEGIN LATEX
\begin{sloppy}
%END LATEX
Here are three possible default rules you may wish to use.  Other rules are
possible but are not as useful.
\begin{itemize}
\item
  \code{@\refclass{nullness/quals}{Nullable}}:  Unannotated types are regarded as possibly-null, or
  nullable.  This default is backward-compatible with Java, which permits
  any reference to be null.  You can activate this default by writing
  a \code{@DefaultQualifier("Nullable")} annotation on a
  % package/
  class or method
  % /variable
  declaration.
\item
  \code{@\refclass{nullness/quals}{NonNull}}:  Unannotated types are treated as non-null.
  % This may leads to fewer annotations written in your code.
  You can activate this
  default via the
  \code{@DefaultQualifier("NonNull")} annotation.
\item
  Non-null except locals (NNEL):  Unannotated types are treated as
  \code{@\refclass{nullness/quals}{NonNull}}, \emph{except} that the
  unannotated raw type of a local variable is treated as
  \code{@\refclass{nullness/quals}{Nullable}}.  (Any generic arguments to a
  local variable still default to
  \code{@\refclass{nullness/quals}{NonNull}}.)  This is the standard
  behavior.  You can explicitly activate this default via the
  \code{@DefaultQualifier(value="NonNull",
    locations=\discretionary{}{}{}\{DefaultLocation\discretionary{}{}{}.ALL\_EXCEPT\_LOCALS\})}
  annotation.

  The NNEL default leads to the smallest number of explicit annotations in
  your code~\cite{PapiACPE2008}.  It is what we recommend.  If you do not
  explicitly specify a different default, then NNEL is the default.
\end{itemize}
%BEGIN LATEX
\end{sloppy}
%END LATEX

\subsection{Conditional nullness\label{conditional-nullness}}

The Nullness Checker supports a form of conditional nullness types, via the
\code{@\refclass{nullness/quals}{AssertNonNullIfTrue}} and \code{@\refclass{nullness/quals}{AssertNonNullIfFalse}} method annotations.
The annotation on a method declares that some expressions are non-null, if
the method returns true (false, respectively).

Consider \sunjavadoc{java/io/File.html}{java.io.File}.
Method
\sunjavadoc{java/io/File.html#listFiles()}{File.listFiles()} may
return null, but is specified to return a non-null value if
\sunjavadoc{java/io/File.html#isDirectory()}{File.isDirectory()} is
true.  The same holds for method
\sunjavadoc{java/io/File.html#listFiles()}{File.list()}.
You could declare this relationship in the following way (this particular
example is already
done for you in the annotated JDK that comes with the Checker Framework):

\begin{Verbatim}
  class File {

    @AssertNonNullIfTrue({"list()", "listFiles()"})
    public boolean isDirectory() { ... }

    public File @Nullable [] listFiles();
  }
\end{Verbatim}

A client that checks that a \code{File} reference is indeed that of a directory,
can then de-reference \code{File.isDirectory} safely without any nullness check.

\begin{Verbatim}
  static void analyze(File file) {
    if (file.isDirectory()) {
      for (File child : file.listFiles()) {  // no possible null dereference
        analyze(child);
      }
    } else {
        ... analyze file ...
    }
  }
\end{Verbatim}


\subsection{Inference of \code{@NonNull} and \code{@Nullable} annotations\label{nullness-inference}}

It can be tedious to write annotations in your code.  Tools exist that
can automatically infer annotations and insert them in your source code.
(This is different than type qualifier refinement for local variables
(Section~\ref{type-refinement}), which infers a more specific type for
local variables and uses them during type-checking but does not insert them
in your source code.  Type qualifier refinement is always enabled, no
matter how annotations on signatures got inserted in your source code.)

Your choice of tool depends on what default annotation (see
Section~\ref{null-defaults}) your code uses.  You only need one of these tools.

\begin{itemize}

\item
  Inference of \code{@\refclass{nullness/quals}{Nullable}}:
  %
  If your code uses the standard NNEL (non-null-except-locals) default or
  the \refclass{nullness/quals}{NonNull} default, then use the
  \ahref{http://groups.csail.mit.edu/pag/daikon/download/doc/daikon.html#AnnotateNullable}{AnnotateNullable}
  tool of the \ahref{http://pag.csail.mit.edu/daikon/}{Daikon invariant
    detector}.

\item
  Inference of \code{@\refclass{nullness/quals}{NonNull}}:
  %
  If your code uses the Nullable default, use one of these tools:
\begin{itemize}
\item
  \ahref{http://julia.scienze.univr.it/}{Julia analyzer},
\item
  \ahref{http://nit.gforge.inria.fr}{Nit: Nullability Inference Tool},
\item
  \ahref{http://jastadd.org/jastadd-tutorial-examples/non-null-types-for-java/}{Non-null
    checker and inferencer} of the \ahref{http://jastadd.org/}{JastAdd
    Extensible Compiler}.
\end{itemize}

\end{itemize}



\section{What the Nullness checker checks\label{nullness-checks}}

The checker issues a warning in three cases:

\begin{enumerate}

\item
  When an expression of non-\code{@\refclass{nullness/quals}{NonNull}} type
  is dereferenced, because it might cause a null pointer exception.
  Dereferences occur not only when a field is accessed, but when an array
  is indexed, an exception is thrown, a lock is taken in a synchronized
  block, and more.  For a complete description of all checks performed by
  the Nullness checker, see the Javadoc for
  \refclass{nullness}{NullnessVisitor}.

\item
  When an expression of \code{@\refclass{nullness/quals}{NonNull}} type
  might become null, because it
  is a misuse of the type:  the null value could flow to a dereference that
  the checker does not warn about.

\item
  \label{lint-nulltest}
  When a null check is performed against a value that is guaranteed to be
  non-null, as in \code{("m" == null)}, because this might indicate a
  programmer error or misunderstanding, and is unnecessary.  This check is
  performed only if the \code{nulltest} lint option is enabled via the
  \code{-Alint=nulltest} command-line option.  The lint option is disabled
  by default because sometimes such checks are part of ordinary defensive
  programming.  See Section~\ref{suppressing-warnings} for more details
  about the \code{-Alint} command-line option.

\end{enumerate}

This example illustrates the programming errors that the checker detects:

\begin{Verbatim}
           Object   obj;  // might be null
  @NonNull Object nnobj;  // never null
  ...
  obj.toString()         // checker warning:  dereference might cause null pointer exception
  nnobj = obj;           // checker warning:  nnobj may become null
  if (nnobj == null)     // checker warning:  redundant test
\end{Verbatim}

Parameter passing and return values are checked analogously to assignments.

The Nullness Checker also checks the correctness, and correct use, of
rawness annotations for checking initialization.  See
Section~\ref{raw-partially-initialized}.


\section{Suppressing nullness warnings\label{suppressing-warnings-nullness}}

The Checker Framework supplies several ways to suppress warnings, most
notably the \<@SuppressWarnings("nullness")> annotation (see
Section~\ref{suppressing-warnings}).  An example use is

\begin{Verbatim}
    // might return null
    @Nullable Object getObject() { ... }

    void myMethod() {
      // The programmer knows that this partucular call never returns null.
      @SuppressWarnings("nullness")
      @NonNull Object o2 = getObject();
\end{Verbatim}


The Nullness Checker supports an additional warning suppression key,
\<nullness:collection-typeargs>.
Use of \<@SuppressWarnings("nullness:generic.argument")> causes the Nullness
Checker to suppress warnings related to misuse of generic type
arguments.  One use for this key is when a class is declared to take only
\<@NonNull> type arguments, but you want to instantiate the class with a
\<@Nullable> type argument, as in \code{List<@Nullable Object>}.  For a more
complete explanation of this example, see
Section~\refwithpage{faq-list-map-nonnull-typeargs}.

The Nullness Checker also permits you to use assertions or method calls to
suppress warnings; see below.

% TODO: check whether the SuppressWarnings keys are correct.


\subsection{Suppressing warnings with assertions and method calls\label{suppressing-warnings-with-assertions}}

Occasionally, it is inconvenient or
verbose to use the \<@SuppressWarnings> annotation.  For example, Java does
not permit annotations such as \<@SuppressWarnings> to appear on statements.

For situations when the \<@SuppressWarnings> annotation is inconvenient,
the Nullness Checker provides three additional ways to suppress warnings:
via an \<assert> statement, the \<castNonNull> method, and the
\<@AssertParametersNonNull> annotation.  These are
appropriate when the Nullness Checker issues a warning, but the programmer
knows for sure that the warning is a false positive, because the value
cannot ever be null at run time.

  % "@SuppressWarnings(nullness)" might be a better string.
  % That enables a programmer to search the source code to find all instances.
\newcommand{\nullnessSuppressionString}{nullness}

\begin{enumerate}
\item
  Use an assertion.  If the string ``\<\nullnessSuppressionString>''
  appears in the message body, then the Nullness Checker treats the
  assertion as suppressing a warning and assumes that the assertion always
  succeeds.  For example, the checker assumes that no null pointer
  exception can occur in code such as
\begin{Verbatim}
  assert x != null : "@SuppressWarnings(nullness)";
  ... x.f ...
\end{Verbatim}

  If the string ``\<\nullnessSuppressionString>'' does not appear in the
  assertion message, then the Nullness Checker treats the assertion as being
  used for defensive programming, and it warns if the method might throw a
  nullness-related exception.

  A downside of putting the string in the assertion message is that if the
  assertion ever fails, then a user might see the string and be confused.
  But the string should only be used if the programmer has reasoned that
  the assertion can never fail.

% (Another way of stating the Nullness checker's use of assertions is as an
% additional caveat to the guarantees provided by a checker
% (Section~\ref{checker-guarantees}).  The Nullness checker prevents null
% pointer errors in your code under the assumption that assertions are
% enabled, and it does not guarantee that all of your assertions succeed.)


% TODO:
% There is a new lint option "flow:inferFromAsserts" that can be used
% to always infer information from assert statements, even without the
% suppression key.
% Discuss how this influences the guarantees from the checker.
% This is a feature of the framework, not only the nullness checker;
% however, the nullness checker is probably the only one that infers something
% from conditions.
% Move the discussion somewhere else, but keep something here?
% Maybe warnings or advanced-features.

% TODO: Also discuss the whole inference over conditions somewhere.

\item
  Use the \refmethod{nullness}{NullnessUtils}{castNonNull}{(T)} method.

The Nullness
 Checker considers both the return value, and also the argument, to
 be non-null after the method call.  Therefore, the
 \<castNonNull> method can be used either as a cast expression or
 as a statement.  The Nullness Checker issues no warnings in any of
the following code:

\begin{Verbatim}
  // one way to use as a cast:
  @NonNull String s = castNonNull(possiblyNull1);

  // another way to use as a cast:
  castNonNull(possiblyNull2).toString();

  // one way to use as a statement:
  castNonNull(possiblyNull3);
  possiblyNull3.toString();`
\end{Verbatim}

  The method also throws \<AssertionError> if Java assertions are enabled and
  the argument is \<null>.  However, it is not intended for general defensive
  programming; see Section~\ref{defensive-programming}.

  A potential disadvantage of using the \<castNonNull> method is that your
  code becomes dependent on the Checker Framework at run time as well as at
  compile time.  You can avoid this by copying the implementation of
  \<castNonNull> into your own code, and possibly renaming it if you do not
  like the name.  Be sure to retain the documentation that indicates that
  your copy is intended for use only to suppress warnings and not for
  defensive programming.  See Section~\ref{defensive-programming} for an
  explanation of the distinction.

\item
  Use the \code{@\refclass{nullness/quals}{AssertParametersNonNull}}
  annotation.  It is used on \<castNonNull>, and may be used on other
  methods with the same semantics; it should probably never be used in any
  other situation.

\end{enumerate}


\subsection{Suppressing warnings on nullness-checking routines and defensive programming\label{defensive-programming}}

%% Work this in
% As explained in Section~\ref{annotate-normal-behavior}, annotations should
% indicate normal behavior that will not cause an exception.
%
% TODO: discuss how to write your own, and why the default doesn't have
% assert or checking methods suppress warnings.


One way to suppress warnings in the Nullness Checker is to use
method \code{castNonNull}.
(Section~\ref{suppressing-warnings-with-assertions} gives other techniques.)

This section explains why the Nullness Checker introduces a new method
rather than re-using the \<assert> statement (as in
\<assert x != null>) or an existing method such as:

\begin{Verbatim}
  org.junit.Assert.assertNotNull(Object)
  com.google.common.base.Preconditions.checkNotNull(Object)
\end{Verbatim}

In each case, the assertion or method indicates an application invariant --- a
fact that should always be true.  There are two distinct reasons a
programmer may have written the invariant, depending on whether the
programmer is 100\% sure that the application invariant holds.

\begin{enumerate}
\item
  A programmer might write it as \textbf{defensive programming}.  This causes
  the program to throw an exception, which is useful for debugging because
  it gives an earlier run-time indication of the error.
  A programmer would use an assertion in this way if the programmer is not
  100\% sure that the application invariant holds.

  % , or even to document what the program
  % is intended to do.

\item
  A programmer might write it to \textbf{suppress} false positive
  \textbf{warning messages} from a checker.  A programmer would use an
  assertion this way if the programmer is 100\% sure that the application
  invariant holds, and the reference can never be null at run time.

\end{enumerate}

With assertions and existing methods like JUnit's \<assertNotNull>, there
is no way of knowing the programmer's intent in using the method.
Different programmers or codebases may use them in different ways.
Guessing wrong would make the Nullness Checker less useful, because it
would either miss real errors or issue warnings where there is no real
error.  Also, different checking tools issue different false warnings that
need to be suppressed, so warning suppression needs to be customized for
each tool rather than inferred from general-purpose code.


As an example of using assertions for defensive programming, some style
guides suggest using assertions or method calls to indicate nullness.  A
programmer might write

\begin{Verbatim}
    String s = ...
    assert s != null;    // or:  assertNotNull(s);   or: checkNotNull(s);
    ... Double.valueOf(s) ...
\end{Verbatim}

A programming error might cause \<s> to be null, in which case the code
would throw an exception at run time.
If the assertion caused the Nullness Checker to assume that \<s> is not
\<null>, then the Nullness Checker would issue no warning for this code.
That would be undesirable, because the whole purpose of the Nullness
Checker is to give a compile-time warning about possible run-time
exceptions.  Furthermore, if the programmer uses assertions for defensive
programming systematically throughout the codebase, then many useful
Nullness Checker warnings would be suppressed.


Because it is important to distinguish between the two uses of assertions
(defensive programming vs.~suppressing warnings), the Checker Framework
introduces the \refmethod{nullness}{NullnessUtils}{castNonNull}{(T)} method.
Unlike existing assertions and
methods, \<castNonNull> is intended only to suppress false warnings that are
issued by the Nullness Checker, not for defensive programming.

If you know that a particular codebase uses
% the \<assert> statement or
a nullness-checking method not for defensive programming but to indicate
facts that are guaranteed to be true (that is, these assertions will never
fail at run time), then you can cause the Nullness Checker to suppress
warnings related to them, just as it does for \<castNonNull>.
Annotate its definition just as
\refmethod{nullness}{NullnessUtils}{castNonNull}{(T)} is annotated (see the
source code for the Checker Framework).
% TODO:
% For an assert statement, XXXXX.
Also, be sure to document the intention in the method's Javadoc, so that
programmers do not
accidentally misuse it for defensive programming.


If you are annotating a codebase that already contains precondition checks,
such as:

\begin{Verbatim}
  public String get(String key, String def) {
    checkNotNull(key, "key"); //NOI18N
    ...
  }
\end{Verbatim}

\noindent
then you should mark the appropriate parameter as \<@NonNull> (which is the
default).  This will prevent the checker from issuing a warning about the
\<checkNotNull> call.


\section{\code{@Raw} annotation for partially-initialized objects\label{raw-partially-initialized}}

The rawness hierarchy indicates whether an object is fully initialized ---
that is, whether its fields have all been assigned.  This is mostly
relevant within the constructor, or for references to \code{this} that
escape the constructor.  Most readers can skip this section on first
reading; you can return to it once you have mastered the rest of the
nullness checker.

The rawness hierarchy is independent of the
nullness hierarchy, and is shown in Figure~\ref{fig:nonnull-hierarchy}.
The rawness hierarchy contains these qualifiers:

\begin{description}

\item[\code{@\refclass{nullness/quals}{Raw}}]
  indicates a type that contains a partially-initialized object.  In a
  partially-initialized object, fields that are annotated as
  \code{@\refclass{nullness/quals}{NonNull}} may be null because the field
  has not yet been assigned.  Within the constructor,
  \code{this} has \code{@\refclass{nullness/quals}{Raw}} type until all
  the fields have been assigned.

% Cut this?
\item[\code{@\refclass{nullness/quals}{NonRaw}}]
  indicates a type that contains a fully-initialized object.  \code{NonRaw}
  is the default, so there is little need for a programmer to write this
  explicitly.

\item[\code{@\refclass{nullness/quals}{PolyRaw}}]
  indicates qualifier polymorphism over rawness (see
  Section~\ref{qualifier-polymorphism}).

\end{description}

Suppose a class contains a field ``\code{@\refclass{nullness/quals}{NonNull}
  Date d;}''.  Java executes the class's constructor by first setting \<d> to
\code{null}.  The constructor sets field \<d> to its final value, either
directly or by calling helper methods.  Before the constructor sets field
\<d>, its initial value \code{null} violates its type \code{@NonNull Date}.
In general, code can depend on field \<d> not being null, but not in a
partially-initialized object.  A partially-initialized object (\code{this}
in a constructor) may be passed to a helper method or stored in a variable;
if so, the method receiver, or the field, would have to be annotated as
\<@Raw>.

% However, if the constructor makes
% a method call (passing \code{this} as a parameter or the receiver), then
% the called method could observe the object in an illegal state.

The \code{@\refclass{nullness/quals}{Raw}} type annotation represents a
partially-initialized object.  If a reference has
\code{@Raw} type, then all of its \code{@NonNull} fields are treated as
\code{@\refclass{nullness/quals}{LazyNonNull}}:  when read, they are
treated as being \code{@\refclass{nullness/quals}{Nullable}}, but when
written, they are treated as being
\code{@\refclass{nullness/quals}{NonNull}}.


The rawness hierarchy is orthogonal to the nullness hierarchy.  It
is legal for a reference to be \<@NonNull @Raw>, \<@Nullable @Raw>,
\<@NonNull @NonRaw>, or \<@Nullable @NonRaw>.  The nullness hierarchy tells
you about the reference itself:  might the reference be null?  The rawness
hierarchy tells you about the \<@NonNull> fields in the referred-to object:
might those fields be temporarily null in contravention of their
declaration?


% Does our implementation handle static fields soundly?


You can suppress warnings related to partially-initialized objects with
\<@SuppressWarnings("rawness")>.  (Do not confuse this with the unrelated
\<@SuppressWarnings("rawtypes")> annotation for non-instantiated generic types!)


\paragraph{How an object becomes non-raw}

Within the constructor,
\code{this} starts out with \code{@\refclass{nullness/quals}{Raw}} type.
As soon as all of the \code{@\refclass{nullness/quals}{NonNull}} fields
have been initialized, then \code{this} is treated as non-raw.

The Nullness checker issues an error if the constructor fails to initialize
any non-null field.  This ensures that the object is in a legal (non-raw)
state by the time that the constructor exits.
\urldef{\jlsdefiniteassignmenturl}{\url}{http://java.sun.com/docs/books/jls/third_edition/html/defAssign.html}
\urldef{\jlsfinalvariablesurl}{\url}{http://java.sun.com/docs/books/jls/third_edition/html/typesValues.html#4.12.4}
This is different than Java's test for definite assignment (see
\ahref{\jlsdefiniteassignmenturl}{JLS ch.16}),
% , which requires that local
% variables (and blank \code{final} fields) must be assigned.  Java does not
% require that non-\code{final} fields be assigned, since
which does not apply to fields (except blank final ones, defined in
\ahref{\jlsdefiniteassignmenturl}{JLS \S 4.12.4}) because fields
have a default value of null.


% and can only be passed to methods when the corresponding parameter is
% annotated with \code{@\refclass{nullness/quals}{Raw}}.  Similar
% restrictions apply to assigning \code{this} to a field.

Currently, the type-checker requires that all fields either have a a
default in the field declaration, or be initialized in the constructor.  If
your code initializes (some) fields in a helper method, you will need to
suppress some warnings.


\paragraph{Invoking the superclass constructor; rawness of the superclass reference}

Suppose that class B extends class A.  Within the B
constructor, until the A superclass constructor is called, \code{this} has
type \code{@Raw B} and also \code{@Raw A}.  After the
superclass constructor has been exited, then \code{this} has type
\code{@Raw B} and also \code{@NonRaw A}.
By the time that the constructor exits, \<this> has type \<@NonRaw B> and
also \<@NonRaw A>.

When you write \<@Raw>, the annotation applies only to the given class, not
to any superclass.  For instance, the checker interprets \<@Raw B> as
``\code{@Raw B} and also \code{{\bfseries @NonRaw} A}'', rather than
``\code{@Raw B} and also \code{@Raw A}'', which would be less useful.  The
only exception is when a method overriding relationship forces the
superclass to also be raw.  For example:

\begin{Verbatim}
  class A extends Object {
    // receiver is "@NonRaw A"
    void nonRawAReceiver() { }
    // annotation forces receiver to be "@Raw A"; also is "@NonRaw Object"
    void rawAReciever() @Raw { }
  }

  class B extends A {
    // annotation forces receiver to be "@Raw B", method overriding forces "@Raw A"
    void rawAReceiver() @Raw {
      super.nonRawAReceiver();  // illegal!  rawness of A does not match
    }
    // annotation forces receiver to be "@Raw B"; also is "@NonRaw A"
    void rawBReceiver() @Raw {
      super.nonRawAReceiver();  // OK
    }
  }
\end{Verbatim}




% \urldef{\jlsconstructorbodyurl}{\url}{http://java.sun.com/docs/books/jls/third_edition/html/classes.html#8.8.7}
% (Recall that the superclass constructor is called on the first line, or is
% inserted automatically by the compiler before the first line, see
% \ahref{\jlsconstructorbodyurl}{JLS \S8.8.7}.)



% Should we change the terminology?
\paragraph{A note about the terminology ``raw''}

The name ``raw'' comes from a research paper that proposed this
approach~\cite{FahndrichL2003}.
A better name might have been ``not yet initialized'' or ``partially
initialized'', but the term ``raw'' is now well-known.
The \code{@\refclass{nullness/quals}{Raw}}
annotation has nothing to do with the raw types of Java Generics.


\section{Map key annotations\label{map-keys}}

Java's
\sunjavadoc{java/util/Map.html#get(java.lang.Object)}{\code{Map.get}}
method always has the possibility to return null, if the key is not in the
map.  Thus, to guarantee that the value returned from \code{Map.get} is
non-null, it is necessary that the map contains only non-null values,
\emph{and} the key is in the map.
The \code{@\refclass{nullness/quals}{KeyFor}} annotation states the latter
property.

If a type is annotated as \code{@KeyFor("m")}, then any value v with that type
is a key in Map m.  Another way of saying this is that the expression
\code{m.containsKey(v)} evaluates to true.

You usually do not have to write \code{@KeyFor} explicitly, because the
checker infers it based on usage patterns, such as calls to
\code{containsKey} or iteration over a map's
\sunjavadoc{java/util/Map.html#keySet()}{\textrm{key set}}.

One usage pattern where you \emph{do} have to write \<@KeyFor> is for a
user-managed collection that is a subset of the key set:

\begin{Verbatim}
Map<String, Object> m;
Set<@KeyFor("m") String> matchingKeys; // keys that match some criterion
for (@KeyFor("m") String k : matchingKeys) {
  ... m.get(k) ...  // known to be non-null
}
\end{Verbatim}

As with any annotation, use of the \<@KeyFor> annotation may force you to
slightly refactor your code.  For example, this would be illegal:

\begin{Verbatim}
  Map<K,V> m;
  Collection<@KeyFor("m") K> coll;
  coll.add(x);   // compiler error, because the @KeyFor annotation is violated
  m.put(x, ...);
\end{Verbatim}

\noindent
but this would be OK (no compiler error):

\begin{Verbatim}
  Collection<@KeyFor("m") K> coll;
  m.put(x, ...);
  coll.add(x);
\end{Verbatim}


Because the \<@KeyFor> type hierarchy is independent from the nullness and
rawness hierarchies, it uses a different warning suppression key.
You can suppress warnings related to map keys with
\<@SuppressWarnings("keyfor")>.


\section{Examples\label{nullness-example}}

\subsection{Tiny examples\label{nullness-tiny-examples}}

To try the Nullness checker on a source file that uses the \code{@\refclass{nullness/quals}{NonNull}} qualifier,
use the following command (where \code{javac} is the JSR 308 compiler that
is distributed with the Checker Framework):

\begin{Verbatim}
  javac -processor checkers.nullness.NullnessChecker examples/NullnessExample.java
\end{Verbatim}

\noindent
Compilation will complete without warnings.

To see the checker warn about incorrect usage of annotations (and therefore the
possibility of a null pointer exception at run time), use the following command:

\begin{Verbatim}
  javac -processor checkers.nullness.NullnessChecker examples/NullnessExampleWithWarnings.java
\end{Verbatim}

\noindent
The compiler will issue three warnings regarding violation of the semantics of
\code{@\refclass{nullness/quals}{NonNull}}.
% in the \code{NonNullExampleWithWarnings.java} file.


\subsection{Annotated library\label{nullness-annotated-library}}

Some libraries that are annotated with nullness qualifiers are:

\begin{itemize}
\item
The Nullness checker itself.

\item
The
\ahref{http://code.google.com/p/plume-lib/}{Plume-lib library}.
Run the command \code{make check-nullness}.


\item
The
\ahref{http://groups.csail.mit.edu/pag/daikon/}{Daikon invariant detector}.
Run the command \code{make check-nullness}.

% \item
% The annotation scene library.
% To run the Nullness checker on the annotation scene library,
% % TODO: how does one do this?
% first download the scene library suite (which includes build
% dependencies for the scene library as well as its source code) and extract it
% into your checkers installation. The checker can then be run on the annotation
% scene library with Apache Ant using the following commands:
%
% \begin{Verbatim}
%   cd checkers
%   ant -f scene-lib-test.xml
% \end{Verbatim}
%
% % \noindent
% % where \code{checkers} is the location of the Checker Framework installation.
%
% You can view the annotated source code, which contains \code{@\refclass{nullness/quals}{NonNull}} annotations, in
% the
% %BEGIN LATEX
% \begin{smaller}
% %END LATEX
% \code{checkers/scene-lib-test/src/annotations/}
% %BEGIN LATEX
% \end{smaller}
% %END LATEX
% directory.

\end{itemize}


\section{Other tools for nullness checking\label{nullness-related-work}}

\newcommand{\linktoNonNull}{\code{\refclass{nullness/quals}{NonNull}}}
\newcommand{\linktoNullable}{\code{\refclass{nullness/quals}{Nullable}}}

The Checker Framework's nullness annotations are similar to annotations used
in IntelliJ IDEA, FindBugs, JML, the JSR 305 proposal, NetBeans, and other tools.  Also
see Section~\ref{other-tools} for a comparison to other tools.

You might prefer to use the Checker Framework because it has a more
powerful analysis that can warn you about more null pointer errors in your
code.

If your code is already annotated with a different nullness
annotation, you can reuse that effort by converting them to the Checker
Framework's nullness annotations.  Perform the refactoring described in
Figure~\ref{fig:nullness-refactoring}.


% These lists should be kept in sync with NullnessAnnotatedTypeFactory.java .
\begin{figure}
\begin{center}
% The ~ around the text makes things look better in Hevea (and not terrible
% in LaTeX).
\begin{tabular}{ll}
\begin{tabular}{|l|}
\hline
 ~com.sun.istack.NotNull~ \\ \hline
 ~edu.umd.cs.findbugs.annotations.NonNull~ \\ \hline
 ~javax.annotation.Nonnull~ \\ \hline
%eventually: ~org.eclipse.jdt.annotation.NonNull~ \\ \hline
 ~org.jetbrains.annotations.NotNull~ \\ \hline
 ~org.netbeans.api.annotations.common.NonNull~ \\ \hline
 ~org.jmlspecs.annotation.NonNull~ \\ \hline
\end{tabular}
&
$\Rightarrow$
~checkers.nullness.quals.NonNull~
\\
\
\\
\begin{tabular}{|l|l|}
\hline
 ~com.sun.istack.Nullable~ \\ \hline
 ~edu.umd.cs.findbugs.annotations.Nullable~ \\ \hline
 ~edu.umd.cs.findbugs.annotations.CheckForNull~ \\ \hline
 ~edu.umd.cs.findbugs.annotations.UnknownNullness~ \\ \hline
 ~javax.annotation.Nullable~ \\ \hline
 ~javax.annotation.CheckForNull~ \\ \hline
 ~javax.validation.constraints.NotNull~ \\ \hline
%eventually: ~org.eclipse.jdt.annotation.Nullable~ \\ \hline
 ~org.jetbrains.annotations.Nullable~ \\ \hline
 ~org.netbeans.api.annotations.common.CheckForNull~ \\ \hline
 ~org.netbeans.api.annotations.common.NullAllowed~ \\ \hline
 ~org.netbeans.api.annotations.common.NullUnknown~ \\ \hline
 ~org.jmlspecs.annotation.Nullable~ \\ \hline
\end{tabular}
&
$\Rightarrow$
~checkers.nullness.quals.Nullable~
\end{tabular}
\end{center}
%BEGIN LATEX
\vspace{-1.5\baselineskip}
%END LATEX
\caption{Refactoring for converting other nullness annotations to the
  Checker Framework's annotations.}
\label{fig:nullness-refactoring}
\end{figure}

%% Removed, because it's tedious and should be obvious to a decent programmer.
% Your IDE may be able to do that for you.  Alternately, do the following:
% \begin{enumerate}
% \item
%   replace \<@Nonnull> by \<@NonNull> (note capitalization difference)
% \item
%   replace \<@CheckForNull> by \<@Nullable>
% \item
%   replace \<@UnknownNullness> by \<@Nullable>
% \item
%   convert each single-type import statement (without a ``\<*>'' character)
%    according to the table above.
% \item
%   convert each on-demand import statements, such as ``\<import
%    edu.umd.cs.findbugs.annotations.*;>''.
% \begin{itemize}
%    \item
%   One approach is to change it into a set of single-type imports,
%       then convert the relevant ones.
%    \item
%   Another approach is to change it according to the table above, then
%       try to compile and re-introduce the single-type imports as necessary.
% \end{itemize}
%    These approaches let you continue to use other annotations in the
%    \<edu.umd.cs.findbugs.annotations> package, even though you are not using
%    its nullness annotations.
% \end{enumerate}


Alternately, the Checker Framework can process those other annotations (as
well as its own, if they also appear in your program).  The Checker
Framework has its own definition of the annotations on the left side of
Figure~\ref{fig:nullness-refactoring}, so that they can be used as type
qualifiers.  The Checker Framework interprets them according to the right
side of Figure~\ref{fig:nullness-refactoring}.

The Checker Framework may issue more or fewer errors than another tool.
This is expected, since each tool uses a different analysis.  Remember that
the Checker Framework aims at soundness:  it aims to never miss a possible
null dereference, while at the same time limiting false reports.  Also,
note FindBugs's non-standard meaning for \<@Nullable>
(Section~\ref{findbugs-nullable}).

Because some of the names are the same (\<NonNull>, \<Nullable>), you can
import at most one of the annotations with
conflicting names; the other(s) must be written out fully rather than
imported.


\subsection{Which tool is right for you?\label{choosing-nullness-tool}}

Different tools are appropriate in different circumstances.  Here is a
brief comparison with FindBugs, but similar points apply to other tools.

Checker Framework has a more powerful nullness analysis; FindBugs misses
some real
errors.  However, FindBugs does not require you to annotate your code as
thoroughly as the Checker Framework does.  Depending on the importance of
your code, you may wish to do no nullness checking; the cursory checking of
FindBugs; or the thorough checking of the Checker Framework.  You might
even want to ensure that both tools run, for example if your coworkers or
some other organization are still using FindBugs.  If you know that you
will eventually want to use the Checker Framework, there is no point using
FindBugs first; it is easier to go straight to using the Checker Framework.

FindBugs can find other errors in addition to nullness errors; here
we focus on its nullness checks.  Even if you use FindBugs for its other
features, you may want to use the Checker Framework for analyses that can
be expressed as pluggable type-checking, such as detecting nullness errors.

Regardless of whether you wish to use the FindBugs nullness analysis, you
may continue running all of the other FindBugs analyses at the same time as
the Checker Framework; there are no interactions among them.

If FindBugs (or any other tool) discovers a nullness error that the Checker
Framework does not, please report it to us (see
Section~\ref{reporting-bugs}) so that we can enhance the Checker Framework.



\subsection{Incompatibility note about FindBugs \tt{@Nullable}\label{findbugs-nullable}}

FindBugs has a non-standard definition of \<@Nullable>.  FindBugs's treatment is not
documented in its own
\ahref{http://findbugs.sourceforge.net/api/edu/umd/cs/findbugs/annotations/Nullable.html}{Javadoc};
it is different from the definition of \<@Nullable> in every other tool for
nullness analysis; it means the same thing as \<@NonNull> when applied to a
formal parameter; and it invariably surprises programmers.  Thus, FindBugs's
\<@Nullable> is detrimental rather than useful as documentation.
In practice, your best bet is to not rely on FindBugs for nullness analysis,
even if you find FindBugs useful for other purposes.

You can skip the rest of this section unless you wish to learn more details.

FindBugs suppresses all warnings at uses of a \<@Nullable> variable.
(You have to use \<@CheckForNull> to
indicate a nullable variable that FindBugs should check.)  For example:

\begin{Verbatim}
     // declare getObject() to possibly return null
     @Nullable Object getObject() { ... }

     void myMethod() {
       @Nullable Object o = getObject();
       // FindBugs issues no warning about calling toString on a possibly-null reference!
       o.toString();
     }
\end{Verbatim}

\noindent
The Checker Framework does not emulate this non-standard behavior of
FindBugs, even if the code uses FindBugs annotations.

With FindBugs, you annotate a declaration, which suppresses checking at
\emph{all} client uses, even the places that you want to check.
It is better to suppress warnings at only the specific client uses
where the value is known to be non-null; the Checker Framework supports
this, if you write \<@SuppressWarnings> at the client uses.
The Checker Framework also supports suppressing checking at all client uses,
by writing a \<@SuppressWarnings> annotation at the declaration site.
Thus, the Checker Framework supports both use cases, whereas FindBugs
supports only one and gives the programmer less flexibility.

In general, the Checker Framework will issue more warnings than FindBugs,
and some of them may be about real bugs in your program.
See Section~\ref{suppressing-warnings-nullness} for information about
suppressing nullness warnings.

(FindBugs made a poor choice of names.  The choice of names should make a
clear distinction between annotations that specify whether a reference is
null, and annotations that suppress false warnings.  The choice of names
should also have been consistent for other tools, and intuitively clear to
programmers.  The FindBugs choices make the FindBugs annotations less
helpful to people, and much less useful for other tools.  As a separate
issue, the FindBugs
analysis is also very imprecise.  For type-related analyses, it is best to
stay away from the FindBugs nullness annotations, and use a more capable
tool like the Checker Framework.)



% As background, here is an explanation of the (sometimes surprising)
% semantics of the FindBugs nullness annotations.
%
%  * edu.umd.cs.findbugs.annotations.NonNull     javax.annotation.Nonnull
%    These mean the obvious thing:   the reference is never null.
%
%  * edu.umd.cs.findbugs.annotations.Nullable     javax.annotation.Nullable
%    This means that the value may be null, but that *all warnings should be
%    suppressed* regarding its use.  In other words, the value is really
%    nullable, but clients should treat it as non-null.  For example:
%
%      // declare getObject() to possibly return null
%      @Nullable Object getObject() { ... }
%
%      // FindBugs issues no warning about calling toString on a possibly-null reference
%      getObject().toString();
%
%    In the Checker Framework, this corresponds to declaring the method
%    return value as @Nullable, then suppressing warnings at client uses
%    that are known to be non-null.  An easy way to suppress the warning
%    is by adding an assert statement about the return value.
%
%    (Alternately, you could declare the method return value as @NonNull, and
%    suppress warnings within the method definition where it returns null,
%    but this approach is not recommended because the @NonNull annotation on
%    the return value would be misleading, and warnings should be suppressed
%    at particular sites where they are known to be unnecessary, not at all
%    call sites whatsoever.)
%
%  * edu.umd.cs.findbugs.annotations.CheckForNull      javax.annotation.CheckForNull
%    This means that the value may be null.  To avoid a NullPointerException,
%    every client should check nullness before dereferencing the value.
%    In the Checker Framework, this corresponds to @Nullable.



% LocalWords:  NonNull plugin quals un NonNullExampleWithWarnings java ahndrich
% LocalWords:  NotNull IntelliJ FindBugs Nullable TODO Alint nullable NNEL JSR
% LocalWords:  DefaultLocation Nullness PolyNull nullness AnnotateNullable JLS
% LocalWords:  Daikon JastAdd javac DefaultQualifier boolean MyEnumType NonRaw
% LocalWords:  NullnessAnnotatedTypeFactory NullnessVisitor LazyNonNull PolyRaw
% LocalWords:  inferencer Nonnull CheckForNull UnknownNullness rawtypes de ch
% LocalWords:  castNonNull NullnessUtils assertNotNull codebases checkNotNull
% LocalWords:  Nullability typeargs nulltest AssertNonNullIfTrue listFiles faq
% LocalWords:  isDirectory AssertionError intraprocedurally SuppressWarnings
% LocalWords:  FindBugs's getObject NonNullOnEntry AssertNonNullIfFalse KeyFor
% LocalWords:  AssertParametersNonNull nonnull AssertNonNullAfter ReadOnly
% LocalWords:  keySet
