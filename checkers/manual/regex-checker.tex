\htmlhr
\chapter{Regex checker\label{regex-checker}}

The Regex Checker prevents, at compile-time, use of syntactically invalid
regular expressions.

A regular expression, or regex, is a pattern for matching certain strings
of text.  In Java, a programmer writes a regular expression as a string.
At run time, the string is ``compiled'' into an efficient internal form
(\sunjavadoc{java/util/regex/Pattern.html}{Pattern}) that is used for
text-matching.

The syntax of regular expressions is complex, so it is easy to make a
mistake.  It is also easy to accidentally use a regex feature from another
language that is not supported by Java (see section ``Comparison to Perl
5'' in the \sunjavadoc{java/util/regex/Pattern.html}{Pattern} Javadoc).
Ordinarily, the programmer does not learn of these errors until run time.
The Regex checker warns about these problems at compile time.


\section{Regex annotations\label{regex-annotations}}

The Regex Checker uses one annotation only:
\code{@\refclass{regex/quals}{Regex}}, to indicate valid regular
expression \code{String}s.

The checker implicitly adds the \code{Regex} qualifier to any
\code{String} literal that is a valid regex.


\section{Running the Regex Checker\label{regex-running}}

The Regex Checker can be invoked by running the following command:

\begin{Verbatim}
  javac -processor checkers.regex.RegexChecker MyFile.java ...
\end{Verbatim}

