\htmlhr
\chapter{Handling warnings and legacy code\label{warnings-and-legacy}}

Section~\ref{get-started-with-legacy-code} describes a methodology for
applying annotations to legacy code.  This chapter tells you what to do if,
for some reason, you cannot change your code in such a way as to eliminate
a checker warning.

Also recall that you can convert checker errors into warnings via the
\code{-Awarns} command-line option; see Section~\ref{checker-options}.


\section{Checking partially-annotated programs:  handling unannotated code\label{unannotated-code}}

Sometimes, you wish to type-check only part of your program.
You might focus on the most mission-critical or error-prone part of your
code.  When you start to use a checker, you may not wish to annotate
your entire program right away.
% Not having source code is *not* a reason.
You may not have
enough knowledge to annotate poorly-documented libraries that your program uses.

If annotated code uses unannotated code, then the checker may issue
warnings.  For example, the Nullness Checker (Chapter~\ref{nullness-checker}) will
warn whenever an unannotated method result is used in a non-null context:

\begin{Verbatim}
  @NonNull myvar = unannotated_method();   // WARNING: unannotated_method may return null
\end{Verbatim}

If the call \emph{can} return null, you should fix the bug in your program by
removing the \code{@\refclass{nullness/quals}{NonNull}} annotation in your own program.

If the library call \emph{never} returns null,
there are several ways to eliminate the compiler warnings.
\begin{enumerate}
\item Annotate \code{unannotated\_method} in full.  This approach provides
  the strongest guarantees, but may require you to annotate additional
  methods that \code{unannotated\_method} calls.  See
  Chapter~\ref{annotating-libraries} for a discussion of how to annotate
  libraries for which you have no source code.
\item Annotate only the signature of \code{unannotated\_method}, and
  suppress warnings in its body.  Two ways to suppress the warnings are via a
  \code{@SuppressWarnings} annotation or by not running the checker on that
  file (see Section~\ref{suppressing-warnings}).
\item Suppress all warnings related to uses of \code{unannotated\_method}
  via the \code{skipUses} processor option
  (see Section~\ref{askipuses}).
  Since this can suppress more warnings than you may expect,
  it is usually better to annotate at least the method's signature.  If you
  choose the boundary between the annotated and unannotated code wisely,
  then you only have to annotate the signatures of a limited number of
  classes/methods
  (e.g., the public interface to a library or package).

\end{enumerate}

Chapter~\ref{annotating-libraries} discusses adding annotations to
signatures when you do not have source code available.
Section~\ref{suppressing-warnings} discusses suppressing warnings.


If you annotate a third-party library, please share it with us so that we
can distribute the annotations with the Checker Framework; see
Section~\ref{reporting-bugs}.


\section{Suppressing warnings\label{suppressing-warnings}}

You may wish to suppress checker warnings because of unannotated libraries
or un-annotated portions of your own code, because of application
invariants that are beyond the capabilities of the type system, because of
checker limitations, because you are interested in only some of the
guarantees provided by a checker, or for other reasons.  You can suppress
warnings via
\begin{enumerate}
\item
  the \code{@SuppressWarnings} annotation (Section~\ref{suppresswarnings-annotation}),
\item
  the \code{-AsuppressWarnings} command-line option (Section~\ref{suppresswarnings-command-line}),
\item
  the \code{@AssumeAssertion} comment in an \<assert> statement (Section~\ref{assumeassertion}),
\item
  the \code{-AskipUses} command-line option (Section~\ref{askipuses}),
\item
  the \code{-AskipDefs} command-line option (Section~\ref{askipdefs}),
\item
  the \code{-Alint} command-line option (Section~\ref{alint}),
\item
  not using the \code{-processor} command-line option (Section~\ref{no-processor}), or
\item
  checker-specific mechanisms (Section~\ref{checker-specific-suppression}).
\end{enumerate}

\noindent
We now explain these mechanisms in turn.

% See the @SuppressWarningsKey annotation and the getSuppressWarningsKey method.

\subsection{\code{@SuppressWarnings} annotation\label{suppresswarnings-annotation}}

You can suppress specific errors and warnings by use of the
\code{@SuppressWarnings("\emph{checkername}")} annotation, for example
\code{@SuppressWarnings("interning")} or \code{@SuppressWarnings("nullness")}.
The argument \emph{checkername} is in lower case and is derived from the
way you invoke the checker; for example, if you invoke a checker as
\code{javac -processor MyNiftyChecker ...}, then you would suppress its
error messages with \code{@SuppressWarnings("mynifty")}.  (An exception is
the Basic Checker, for which you use the annotation name; see
Section~\ref{basic-using}).

A \code{@\sunjavadoc{java/lang/SuppressWarnings.html}{SuppressWarnings}}
annotation may be placed on program declarations such as a local
variable declaration, a method, or a class.  It suppresses all warnings
related to the given checker, for that program element.
\<@SuppressWarnings> is a declaration annotation, and it cannot be used on
statements, expressions, or types.

For instance, one common use is
to suppress warnings at a cast that you know is safe.  Here is an example
that uses the Tainting Checker (Section~\ref{tainting-checker}); assume
that \<expr> has type \<@Tainted String>:

\begin{Verbatim}
  @SuppressWarnings("tainting") // Explain why the suppression is sound.
  @Untainted String myvar = expr;
\end{Verbatim}

\noindent
It would have been illegal to write

\begin{Verbatim}
  @Untainted String myvar;
  @SuppressWarnings("tainting") // Explain why the suppression is sound.
  myvar = expr;
\end{Verbatim}

\noindent
because Java does not permit annotations (such as \<@SuppressWarnings>) on
assignments or other statements or expressions.


\subsubsection{Good practices when suppressing warnings}

\paragraph{Suppress warnings in the smallest possible scope}

If a particular expression causes a
false positive warning, you should extract that expression into a local variable
and place a \code{@SuppressWarnings} annotation on the variable
declaration, rather than suppressing warnings for a larger expression or an
entire method body.

%% I'm not sure how this is related to the smallest possible scope.
% As another example, if you have annotated the signatures but not the bodies
% of the methods in a class or package, put a \code{@SuppressWarnings}
% annotation on the class declaration or on the package's
% \code{package-info.java} file.

\paragraph{Justify why the warning is a false positive}

An \<@SuppressWarnings> annotation asserts that the code is actually
correct, even though the type system is unable to prove that the code is
correct.

Whenever you write a \<@SuppressWarnings> annotation, you should also
write, typically on the same line or on the previous line, a code comment
explaining why the code is actually correct.  In some cases you might also
justify why the code cannot be rewritten in a simpler way that would be
amenable to type-checking.

This documentation will help you and others to understand the reason for
the \<@SuppressWarnings> annotation.  It will also help if you decide to
audit your code to verify all the warning suppressions.

\paragraph{Use a specific argument to \code{@SuppressWarnings}}

\label{compiler-message-keys}

The \<@SuppressWarnings> argument string can be of the form
\emph{checkername} or
or \emph{checkername:messagekey}.  The \emph{checkername} part is as
described above.  The \emph{messagekey} part suppresses only
errors/warnings relating to the given message key.  For example,
\code{cast.unsafe} is the key for warnings about an unsafe cast, and
\code{cast.redundant} to the key for warnings about a redundant cast.

Thus, the above example could have been written as any one of the
following, which would have suppressed the specific error:

\begin{Verbatim}
  @SuppressWarnings("tainting")              // suppresses all tainting-related warnings
  @SuppressWarnings("tainting:cast.unsafe")  // suppresses tainting warnings about unsafe casts
  @SuppressWarnings("tainting:cast")         // suppresses tainting warnings about casts
\end{Verbatim}

\noindent
For a list of the message keys, 
see the \code{messages.properties} files in 
%BEGIN LATEX
\\
%END LATEX
\code{checker-framework/checkers/src/checkers/\emph{checkername}/messages.properties}.
Each checker is built on the \code{basetype} checker and inherits its
properties.  Thus, to find all the error keys for a checker, you usually
need to examine its own \code{messages.properties} file and that of
\code{basetype}.

If a checker produces a warning/error and you want to determine its message
key, you can re-run the checker, passing the the \code{-Anomsgtext}
command-line option (Section~\ref{debugging-options}).


\subsection{\code{-AsuppressWarnings} command-line option\label{suppresswarnings-command-line}}

Supplying the \<-AsuppressWarnings> command-line option is equivalent to
writing a \<@SuppressWarnings> annotation on every class that the compiler
type-checks.  The argument to \<-AsuppressWarnings> is a comma-separated
list of warning suppression keys, as in
\<-AsuppressWarnings=purity,uninitialized>.

When possible, it is better to write a \<@SuppressWarnings> annotation with a
smaller scope, rather than using the \<-AsuppressWarnings> command-line option.


\subsection{\code{@AssumeAssertion} comment in an \<assert> statement\label{assumeassertion}}

You can suppress a warning by /<assert>ing that some property is true, and
placing the string \<@AssumeAssertion(\emph{warningkey})> in the assertion
message.

For example, in this code:

\begin{Verbatim}
  assert x != null : "@AssumeAssertion(nullness)";
  ... x.f ...
\end{Verbatim}

\noindent
the Nullness Checker assumes that \<x> is non-null from the \<assert>
statement forward, and so the expression \<x.f> cannot throw a null pointer
exception.

The assert expression must be an expression that would affect flow-sensitive
type qualifier refinement (Section~\ref{type-refinement}), if the
expression appeared in a conditional test.  Each type system has its own
rules about what type refinement it performs, if any.

The warning key is exactly as in the \<@SuppressWarnings> annotation
(Section~\ref{suppresswarnings-annotation}).  The same good practices apply
as for a \<@SuppressWarnings> annotations, such as writing a comment
justifying why the assumption is safe.

If the string \<@AssumeAssertion(\emph{warningkey})> does not appear in the
assertion message, then the Checker Framework treats the assertion as
being used for defensive programming.  That is, the programmer believes
that the assertion might fail at run time, so the Checker Framework should
not make any inference, which would not be justified.

A downside of putting the string in the assertion message is that if the
assertion ever fails, then a user might see the string and be confused.
But the string should only be used if the programmer has reasoned that the
assertion can never fail.

% (Another way of stating the Nullness Checker's use of assertions is as an
% additional caveat to the guarantees provided by a checker
% (Section~\ref{checker-guarantees}).  The Nullness Checker prevents null
% pointer errors in your code under the assumption that assertions are
% enabled, and it does not guarantee that all of your assertions succeed.)


% TODO:
% There is a new lint option "flow:inferFromAsserts" that can be used
% to always infer information from assert statements, even without the
% suppression key.
% Discuss how this influences the guarantees from the checker.
% This is a feature of the framework, not only the nullness checker;
% however, the nullness checker is probably the only one that infers something
% from conditions.
% Move the discussion somewhere else, but keep something here?
% Maybe warnings or advanced-features.

% TODO: Also discuss the whole inference over conditions somewhere.



\subsection{\code{-AskipUses} command-line option\label{askipuses}}

You can suppress all errors and warnings at all \emph{uses} of a given
class (but the class itself is still type-checked, unless you also use
the \code{-AskipDefs} command-line option, see~\ref{askipdefs}).
Set the \code{-AskipUses} command-line option to a
regular expression that matches class names (not file names) for which warnings and errors
should be suppressed.  For example, suppose that you use
``{\codesize\verb|-AskipUses=^java\.|}'' on the command line
(with appropriate quoting) when invoking
\code{javac}.  Then the checkers will suppress all warnings related to
classes whose fully-qualified name starts with \codesize\verb|java.|, such
as all warnings relating to invalid arguments and all warnings relating to
incorrect use of the return value.

To suppress all errors and warnings related to multiple classes, you can use
the regular expression alternative operator ``\code{|}'', as in
``{\codesize\verb+-AskipUses="java\.lang\.|java\.util\."+}'' to suppress
all warnings related to uses of classes belong to the \code{java.lang} or
\code{java.util} packages.


\subsection{\code{-AskipDefs} command-line option\label{askipdefs}}

You can suppress all errors and warnings in the \emph{definition} of a given
class (but uses of the class are still type-checked, unless you also use
the \code{-AskipUses} command-line option, see~\ref{askipuses}).
Set the \code{-AskipDefs} command-line option to a
regular expression that matches class names (not file names) in whose definition warnings and errors
should be suppressed.  For example, if you use
``{\codesize\verb|-AskipDefs=^mypackage\.|}'' on the command line
(with appropriate quoting) when invoking
\code{javac}, then the definitions of 
classes whose fully-qualified name starts with \codesize\verb|mypackage.|
will not be checked.

Another way not to type-check a file is not to pass it on the compiler
command-line:  the Checker Framework type-checks only files that are passed
to the compiler on the command line, and does not type-check any file that
is not passed to the compiler.  The \code{-AskipDefs} command-line option
is intended for situations in which the build system is hard to understand
or change.  In such a situation, a programmer may find it easier to supply
an extra command-line argument, than to change the set of files that is
compiled.


\subsection{\code{-Alint} command-line option\label{alint}}

\label{lint-options}

The \code{-Alint} option enables or disables optional checks, analogously to
javac's \code{-Xlint} option.
Each of the distributed checkers supports at least the following lint options:

\begin{itemize}

\item
  \code{cast:unsafe} (default: on) warn about unsafe casts that are not
  checked at run time, as in \code{((@NonNull String) myref)}.  Such casts
  are generally not necessary when flow-sensitive local type refinement is
  enabled.

\item
  \code{cast:redundant} (default: on) warn about redundant
  casts that are guaranteed to succeed at run time,
  as in \code{((@NonNull String) "m")}.  Such casts are not necessary,
  because the target expression of the cast already has the given type
  qualifier.

\item
  \code{cast} Enable or disable all cast-related warnings.

\item
  \code{all} Enable or disable all lint warnings, including
  checker-specific ones if any.  Examples include \code{nulltest} for the
  Nullness Checker (see Section~\ref{lint-nulltest}) and \<dotequals> for
  the Interning Checker (see Section~\ref{lint-dotequals}).  This option
  does not enable/disable the checker's standard checks, just its optional
  ones.

\item
  \code{none} The inverse of \<all>:  disable or enable all lint warnings,
  including checker-specific ones if any.

\end{itemize}

% This syntax is different from -Xlint that uses a colon instead of an
% equals sign, because javac forces the use of the equals sign.

\noindent
To activate a lint option, write \code{-Alint=} followed by a
comma-delimited list of check names.  If the option is preceded by a
hyphen (\code{-}), the warning is disabled.  For example, to disable all
lint options except redundant casts, you can pass
\code{-Alint=-all,cast:redundant} on the command line.

Only the last \code{-Alint} option is used; all previous \code{-Alint}
options are silently ignored.


\subsection{No \code{-processor} command-line option\label{no-processor}}

You can also compile parts of your code without use of the
\code{-processor} switch to \code{javac}.  No checking is done during
such compilations.

\subsection{Checker-specific mechanisms\label{checker-specific-suppression}}

Finally, some checkers have special rules.  For example, the Nullness
checker (Chapter~\ref{nullness-checker}) uses \code{assert} statements that contain
null checks, and the special \<castNonNull> method, to suppress warnings
(Section~\ref{suppressing-warnings-with-assertions}).
This manual also explains special mechanisms for
suppressing warnings issued by the Fenum checker
(Section~\ref{fenum-suppressing}) and the Units checker
(Section~\ref{units-suppressing}).


\section{Backward compatibility with earlier versions of Java\label{backward-compatibility}}

Sometimes, your code needs to be compiled by people who are not using a
compiler that supports type annotations.
Sections~\ref{annotations-in-comments}--\ref{uncommenting-annotations}
discuss this situation, which you can handle by writing annotations in
comments.

In other cases, your code needs to be run by people who are not using a Java~8
JVM\@.  Section~\ref{java5-class-files} discusses this situation, which
you can handle by passing the \code{-target 5} command-line argument.

(\textbf{Note:} These are features of the Type Annotations compiler that is
distributed along with the Checker Framework.  They are \emph{not}
supported by the mainline OpenJDK compiler.  These features are the key
difference between the Type Annotations compiler and the OpenJDK compiler
on which it is built.)

%   For more details
% about the differences, see file \code{README-jsr308.html} in the Type
% Annotations distribution.


\subsection{Annotations in comments\label{annotations-in-comments}}

A Java 4 compiler does not permit use of
annotations, and a Java 5/6/7 compiler only permits annotations on
declarations (but not on generic arguments, casts, \<extends> clauses, method receiver, etc.).

So that your code can be compiled by any Java compiler (for any version of
the Java language), you may write any single annotation inside a
\code{/*}\ldots\code{*/} Java comment, as in \code{List</*@NonNull*/ String>}.
The Type Annotations compiler treats the code exactly as if you had not written the
\code{/*} and \code{*/}.
In other words, the Type Annotations compiler will recognize the
annotation, but your code will still compile with any other Java compiler.

This feature only works if you provide no \code{-source} command-line
argument to \code{javac}, or if the \code{-source} argument is \code{1.8}
or \code{8}.

In a single program, you may write some annotations in comments, and others
outside of comments.

By default, the compiler ignores any comment that contains spaces at the
beginning or end, or between the \code{@} and the annotation name.
In other words, it reads \code{/*@NonNull*/} as an annotation but ignores
\code{/* @NonNull*/} or \code{/*@ NonNull*/} or \code{/*@NonNull */}.
This
feature enables backward compatibility with code that contains comments
that start with \code{@} but are not annotations.  (The
ESC/Java~\cite{FlanaganLLNSS02}, JML~\cite{LeavensBR2006:JML}, and
Splint~\cite{Evans96} tools all use ``\code{/*@}'' or ``\code{/*~@}'' as a
comment marker.)
Compiler flag
\code{-XDTA:spacesincomments} causes the compiler to parse annotation comments
even when they contain spaces.  You may need to use
\code{-XDTA:spacesincomments} if you use Eclipse's ``Source $>$ Correct
Indentation'' command, since it inserts space in comments.  But the
annotation comments are less readable with spaces, so you may wish to disable
inserting spaces:  in the Formatter preferences, in the Comments tab,
unselect the ``enable block comment formatting'' checkbox.

Compiler flag \code{-XDTA:noannotationsincomments} causes the compiler
to ignore annotation comments.  With this compiler flag the Type
Annotations compiler behaves like a standard Java 8 compiler that does
not support annotations in comments.  If your code already contains
comments of the form \</*@...*/> that look like type annotations, and
you want the Type Annotations compiler not to try to interpret them,
then you can either selectively add spaces to the comments or use
\code{-XDTA:noannotationsincomments} to turn off all annotation
comments.

There is a more powerful mechanism that permits arbitrary code to be
written in a comment that is formatted as ``\code{/*>>>}\ldots\code{*/}'',
with the first three characters of the comment being greater-than signs. As
with annotations in comments, the commented code is ignored by ordinary
compilers but is treated like ordinary code by the UW Type Annotations
compiler.

This mechanism is intended for two purposes.
First, it supports the receiver (\<this> parameter) syntax, as in

\begin{Verbatim}
public boolean getResult(/*>>> @ReadOnly MyClass this*/) { ... }
public boolean getResult(/*>>> @ReadOnly MyClass this, */ String argument) { ... }
\end{Verbatim}

\noindent
for a method that does not modify its receiver.

Second, it can be used for import statements:

\begin{Verbatim}
/*>>>
import checkers.nullness.quals.*;
import checkers.regex.quals.*;
*/
\end{Verbatim}

\noindent
Such a use avoids dependences on qualifiers.  It also eliminates Eclipse
warnings about unused import statements, if all uses of the imported
qualifier are themselves in comments and thus invisible to Eclipse.

A third use is for adding multiple annotations inside one
comment.  However, it is better style to write multiple annotations each
inside its own comment, as in \</*@NonNull*/ /*@Interned*/ String s;>.

It would be possible to abuse the \code{/*>>>...*/} mechanism to inject
code only when using
the Type Annotations compiler.  Doing so is not a sanctioned use of the
mechanism.


\subsection{Implicit import statements\label{implicit-import-statements}}

When writing source code with annotations, it is more convenient to write a
short form such as \code{@NonNull} instead of
\code{@checkers.nullness.quals.NonNull}.  Here are ways to achieve this goal.

\label{jsr308_imports}

\begin{itemize}
\item
The traditional way to do this is to write an import statement like
``\code{import checkers.nullness.quals.*;}''.  This works, but everyone who
compiles the code (no matter what compiler they use, and even if the
annotations are in comments) must have the annotation definitions (e.g.,
the \code{checkers.jar} or \code{checkers-quals.jar} file) on their
classpath.  The reason is that a Java compiler issues an error if an
imported package is not on the classpath.  See Section~\ref{distributing}.

\item
Write an import statement in a comment, just as for annotations in comments:
\begin{Verbatim}
/*>>> import checkers.nullness.quals.*; */
\end{Verbatim}

\item
An alternative is to set the shell environment variable
\code{jsr308\_imports} when you compile the code.
The Type Annotations compiler treats this as if the given packages/classes were
imported, but other compilers
ignore the
\code{jsr308\_imports} environment variable --- they do not need it, since
they do not support annotations in comments.  Thus, your code can compile
whether or not the Type Annotations compiler is being used.

You can specify multiple packages/classes separated by the classpath separator
(same as the file path separator:  \<;> for Windows, and \<:> for Unix and
Mac).  For example, to implicitly import the Nullness and Interning
qualifiers, set \code{jsr308\_imports} to
\code{checkers.nullness.quals.*:checkers.interning.quals.*}.

Three ways to set an environment variable are:
\begin{itemize}
\item
  Set the environment variable in your shell.  For example, in bash, you
  could do \code{export jsr308\_imports='checkers.nullness.quals.*'}.
  This takes effect for all subsequent commands in that shell.  To take
  effect for all shells that you run, put the command in your
  \verb|~/.bashrc| file.
\item
  Set the environment variable for a single command.  For example, in bash
  prefix the \code{javac} command by
  \code{jsr308\_imports='checkers.nullness.quals.*'}.
\item
  Set the environment variable for a single command, via a \code{javac}
  argument.  Use the \code{javac} command-line argument
  \code{-J-Djsr308\_imports='checkers.nullness.quals.*'}.
\end{itemize}

If you issue the javac command from the command line or in a Makefile, you
may need to add quotes (as shown above), to prevent your shell from
expanding the \code{*} character.
If you supply the \code{-J-Djsr308\_imports} argument via an Ant buildfile,
you do not need the extra quoting.

\item
If it is not possible to set the environment variable (for example, Maven
bug \ahref{http://jira.codehaus.org/browse/PLXCOMP-190}{PLXCOMP-190}
makes it impossible to use the \code{-J-Djsr308\_imports} command-line
argument), you can instead use a different javac command-line argument:
\code{-jsr308\_imports ...} or \code{-Djsr308.imports=...} (they have the
same effect).  The same syntax for the packages/classes, and the same
warnings about quoting from the command line, apply as for the
\code{jsr308\_imports} environment variable.

\end{itemize}



\subsection{Migrating away from annotations in comments\label{uncommenting-annotations}}

Suppose that your codebase currently uses annotations in comments, but you
wish to remove the comment characters around your annotations, because in
the future you will use only compilers that support type annotations.
This Unix command removes
the comment characters, for all Java files in the current
working directory or any subdirectory.

[TODO: This doesn't handle the >>> comments.  Adapt it to do so.]

\begin{Verbatim}
   find . -type f -name '*.java' -print \
     | xargs grep -l -P '/\*\s*@([^ */]+)\s*\*/' \
     | xargs perl -pi.bak -e 's|/\*\s*@([^ */]+)\s*\*/|@\1|g'
\end{Verbatim}

You can customize this command:
\begin{itemize}
\item
To process comments with embedded spaces and asterisks, change
two instances of ``\verb|[^ */]|'' to ``\verb|[^/]|''.
\item
To ignore comments with leading or trailing spaces, remove the four
instances of ``\verb|\s*|''.
\item
  To not make backups, remove ``\verb|.bak|''.
\end{itemize}


If your code used implicit import statements
(Section~\ref{implicit-import-statements}), then after uncommenting the
annotations, you may also need to introduce
explicit import statements into your code.


\subsection{Annotations in Java 5 \code{.class} files\label{java5-class-files}}

If you supply the \code{-target 5} command-line argument along with no
\code{-source} argument (or \code{-source 8}, which is equivalent), then the
Type Annotations compiler creates a \code{.class} file that can be run on a
Java 5 JVM, but that contains the type annotations.  (It does not matter
whether the type annotations were written in comments or not.)  The fact
that the \code{.class} file contains the type annotations is useful when
type-checking client code.  If you try to type-check client code against a
library that lacks type annotations, then spurious warnings can result.
So, use of \code{-target 5} gives backward compatibility with earlier JVMs
while still permitting pluggable type-checking.

Ordinary Java compilers do not let you use a \code{-target} command-line
argument with a value less than the \code{-source} argument.

Use of the \code{-source 5} command-line argument produces a \code{.class}
file that does not contain type annotations.  One reason you might want to
periodically compile with the \code{-source 5} argument is to ensure that
your code does not use any Java 8 features other than type annotations in
comments.


% LocalWords:  quals skipUses un AskipUses Alint annotationname javac's Awarns
% LocalWords:  Xlint dotequals castNonNull XDTA spacesincomments Formatter jsr
% LocalWords:  unselect checkbox classpath Djsr bak Nullness nullness java lang
% LocalWords:  checkername util myref nulltest html ESC buildfile mynifty Fenum
% LocalWords:  MyNiftyChecker messagekey basetype uncommenting Anomsgtext '
% LocalWords:  AskipDefs mypackage jsr308 Djsr308 Makefile PLXCOMP expr '
%  LocalWords:  TODO
